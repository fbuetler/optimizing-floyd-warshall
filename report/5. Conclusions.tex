\section{Conclusions}
To summarize, we examined three instances of the generalized Floyd-Warshall algorithm. Respectively, they solve the all-pairs shortest-path, widest-path and transitive closure problems. We implemented various optimizations for each, namely \texttt{FWI}, \texttt{FWT}, and corresponding vectorized versions.
Finally, we implemented an autotuner that finds locally optimal unrolling and tiling factors and generates corresponding code.

Our benchmarks confirm that the optimization techniques for the all-pairs shortest-path problem by Han, Franchetti and Püschel \cite{han06generation} work, and we show that they generalize to the other two examined closed semi-rings of the widest-path and transitive closure instances. Further, we showed that a single autotuner implementation is
able to successfully apply the same optimizations to all three problems.

However, we also found limits to the generality of our autotuner as soon as the data representation changes.
This leads us to conjecture that our autotuner should be easily generalizable to other closed semi-rings over
$\mathbb{R}^\infty$.

\mypar{Future Work}
An obvious next step would be to implement other closed semi-rings in our autotuner. A good choice would be
the unpivoted Gauss-Jordan transform, as it is a semi-ring over $\mathbb{R}\cup\{\text{undefined}\}$. For further
generalization we would likely need an autotuner which is aware of the program structure and is able to apply
transforms based on the structure as needed for different data representations. Ideally, one would implement
an autotuner fully generic over a closed semi-ring.

There are also more optimizations that could be applied to the Floyd-Warshall algorithm. One example is the
doubly tiled version described by Han et al. \cite{han06generation}. Another possible optimization would be
to explore different algorithms for matrix closure. One possible candidate might be R-Kleene 
\cite{dalberto2007r-kleene}.
