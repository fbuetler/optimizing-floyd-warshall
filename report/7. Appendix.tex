\newpage

\appendix

\section{Appendix}

\subsection{Code Samples}\label{app:code}
\begin{listing}[h!]
\caption{An unrolled implementation of the APSP FW instance.}\label{lst:unrolled}
\begin{minted}{c}
int floydWarshall(double *C, int N) {
  for (int k = 0; k < N; k++) {
    for (int i = 0; i < N; i ++) {
      double cik = C[i * N + k];
      for (int j = 0; j < N - 1; j += 2) {
        // load
        double cij0 = C[i * N + j + 0];
        double cij1 = C[i * N + j + 1];

        double ckj0 = C[k * N + j + 0];
        double ckj1 = C[k * N + j + 1];

        // compute 1
        double sum0 = cik + ckj0;
        double sum1 = cik + ckj1;

        // compute 2
        double res0 = MIN(cij0, sum0);
        double res1 = MIN(cij1, sum1);

        // store
        C[i * N + j + 0] = res0;
        C[i * N + j + 1] = res1;
      }
    }
  }
  return 0;
}
\end{minted}
\end{listing}

\subsection{Autotuner Parameter Choices}
The following are the autotuner's choices of unrolling factors and tile sizes, all stemming from the MM instance.
In the tables, N denotes the input size, $L_1$ the tile size, and $U_{i, j, k}$ the unrolling factors of the various loops.

In the case of \texttt{FWT}, the unrolling factors $U_{i, j}$ concern the function \texttt{FWI}, and the unrolling factors $U_{i, j, k}'$ concern the function \texttt{FWIabc}.

\begin{table}[h]
\centering
\begin{tabular}{|c|c|c|}
\hline
N & $U_i$ & $U_j$ \\
\hline
432 & 3 & 1\\
576 & 3 & 1\\
1152 & 3 & 1\\
1728 & 3 & 1\\
2304 & 3 & 1\\
2880 & 3 & 1\\
3456 & 3 & 1\\
4032 & 3 & 1\\
\hline
\end{tabular}
\caption{Parameters for scalar \texttt{FWI}}
\end{table}

\begin{table}[h]
\centering
\begin{tabular}{|c|c|c|c|c|c|c|}
\hline
N & $L_1$ & $U_i$ & $U_j$ & $U_i'$ & $U_j'$ & $U_k'$ \\
\hline
432 & 144 & 3 & 1 & 1 & 9 & 16\\
576 & 576 & 3 & 1 & 1 & 8 & 18\\
1152 & 96 & 3 & 1 & 1 & 8 & 16\\
1728 & 144 & 3 & 1 & 1 & 8 & 18\\
2304 & 96 & 1 & 1 & 1 & 8 & 16\\
2880 & 192 & 2 & 1 & 1 & 6 & 4\\
3456 & 128 & 2 & 1 & 1 & 8 & 16\\
4032 & 224 & 2 & 1 & 1 & 8 & 16\\
\hline
\end{tabular}
\caption{Parameters for scalar \texttt{FWT}}
\end{table}

\begin{table}[h]
\centering
\begin{tabular}{|c|c|c|}
\hline
N & $U_i$ & $U_j$ \\
\hline
432 & 4 & 8\\
576 & 3 & 4\\
1152 & 8 & 4\\
1728 & 9 & 4\\
2304 & 8 & 4\\
2880 & 8 & 4\\
3456 & 4 & 12\\
\hline
\end{tabular}
\caption{Parameters for vectorized \texttt{FWI}}
\end{table}

\begin{table}[h]
\centering
\begin{tabular}{|c|c|c|c|c|c|c|}
\hline
N & $L_1$ & $U_i$ & $U_j$ & $U_i'$ & $U_j'$ & $U_k'$ \\
\hline
432 & 144 & 9 & 4 & 4 & 12 & 16\\
576 & 96 & 8 & 8 & 4 & 16 & 16\\
1152 & 96 & 8 & 4 & 4 & 8 & 16\\
1728 & 144 & 8 & 4 & 4 & 8 & 16\\
2304 & 96 & 4 & 4 & 3 & 16 & 16\\
2880 & 192 & 8 & 8 & 4 & 8 & 16\\
3456 & 128 & 8 & 4 & 4 & 16 & 16\\
4032 & 192 & 8 & 4 & 4 & 8 & 16\\
\hline
\end{tabular}
\caption{Parameters for vectorized \texttt{FWT}}
\end{table}